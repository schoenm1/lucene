%--------------------------------------------------------------------- INFORMATIONEN -----------------------------------------------------------------------------------------------------------------------
%																															|
% 	Definition von globalen Parametern, die im gesamten Dokument verwendet																|
% 	werden k�nnen (z.B auf dem Deckblatt etc.).																						|
% ----------------------------------------------------------------------------------------------------------------------------------------------------------------------------------------------------------------------
\newcommand{\titel}{Information Retrieval}					% Titel des Dokumentes										|
\newcommand{\untertitel}{}				% Untertitel des Dokumentes									|
\newcommand{\art}{Seminararbeit FS 2014}											% Art der Arbeit (Bacholor, Seminar...)							|
%\newcommand{\fachgebiet}{Wirtschaftsinformatik}								% Fachgebiet der Arbeit										|
\newcommand{\autorFirst}{Micha Sch�nenberger}								% 1. Autor													|
%\newcommand{\autorSecond}{Micha Sch�nenberger}							% 2. Autor													|
%\newcommand{\studienbereich}{Wirtschaftsinformatik}							% Studienbereich											|
\newcommand{\matrikelnrFirst}{0000'0000}									% Martikelnummer 1. Autor									|	
\newcommand{\matrikelnrSecond}{0000'0000}								% Martikelnummer 2. Autor									|
\newcommand{\erstgutachter}{Dr. Ruxandra Domenig}									% Erstgutachter der Arbeit									|
%\newcommand{\zweitgutachter}{Dipl.-Inf. Lukas Podolski}						% Zweitgutachter der Arbeit									|
\newcommand{\jahr}{2014}												% Jahr der Arbeit											|
%																															|
% ----------------------------------------------------------------------------------------------------------------------------------------------------------------------------------------------------------------------


%----------------------------------------------------------------- EIGENE DEFINITIONEN --------------------------------------------------------------------------------------------------------------------
%																															|
\newcommand{\todo}[1]{\textbf{\textsc{\textcolor{red}{(TODO: #1)}}}}					%														|
%																															|
\newcommand{\AutorZ}[1]{\textsc{#1}}										%														|
\newcommand{\Autor}[1]{\AutorZ{\citeauthor{#1}}}								%														|
%																															|
\newcommand{\NeuerBegriff}[1]{\textbf{#1}}									%														|
%																															|
\newcommand{\Fachbegriff}[1]{\textit{#1}}										%														|
\newcommand{\Prozess}[1]{\textit{#1}}										%														|
\newcommand{\Webservice}[1]{\textit{#1}}									%														|
%																															|
\newcommand{\Eingabe}[1]{\texttt{#1}}										%														|
\newcommand{\Code}[1]{\texttt{#1}}											%														|
\newcommand{\Datei}[1]{\texttt{#1}}											%														|
%																															|
\newcommand{\Datentyp}[1]{\textsf{#1}}										%														|
\newcommand{\XMLElement}[1]{\textsf{#1}}									%														|
%																															|
%------------------------------------------------------------------------------ INFO --------------------------------------------------------------------------------------------------------------------------------
%																															|
% \newcommand{\Name}[Anzahl]{Definition}																							|
% @name: 			ein Befehlsname																							|
% @Anzahl: 			(optional) eine ganze Zahl zwischen 1 und 9																		|
% @Definition:			eine Befehlsdefinition																						|
%																															|
% ----------------------------------------------------------------------------------------------------------------------------------------------------------------------------------------------------------------------


% ----------------------------------------------------------------------------------------------------------------------------------------------------------------------------------------------------------------------
%																	%														|
% Abk�rzungen mit korrektem Leerraum										%														|					
%																	%														|
\newcommand{\vgl}{Vgl.\ }												%														|					
\newcommand{\ua}{\mbox{u.\,a.\ }}											%														|
\newcommand{\zB}{\mbox{z.\,B.\ }}											%														|					
\newcommand{\bs}{$\backslash$}											%														|
%																	%														|
%																	%														|
% ----------------------------------------------------------------------------------------------------------------------------------------------------------------------------------------------------------------------





