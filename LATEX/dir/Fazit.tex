\chapter{Fazit}
\label{chapter:Fazit}

Die Einarbeitung mit Lucene war nicht immer einfach. Vor allem wegen der verschiedenen Datei-Formate. Die Nicht-Kompatibilit�t des Standard-Indexer von Lucene mit PDF und Office Dokumenten war recht schnell klar. Jedoch wirft Lucene keine Fehlermeldung, sondern indiziert die Dateien, jedoch sinnlos, da der Inhalt falsch interpretiert wird.\\
Bei den Office-Dokumenten war es schwieriger. Anfangs wurde eine Indexer f�r Excel und ein Indexer f�r Word Dokumente eingesetzt. Jedoch wurden mit diesem Indexer nur �ltere Office-Dikumente richtig indiziert. F�r XML-basierte Office-Dokumente musste nachtr�glich nochmals ein eigener Indexer hinzugef�gt werden.

Aufgrund dessen, dass vor Beginn der Arbeit noch nie mit Lucene gearbeitet wurde, war eine Einarbeitungszeit unabdingbar.
Mittels eines Testordners mit verschiedenen Dateiendungen und einem Subfolder wurde die Implementation der Java-Applikation gestestet, was einwandfrei funktionierte. Leider war die Applikation in der Praxis beim Indizieren der Schuldateien nicht perfomant und es wurden hundertaussende Dateien indiziert. Das Problem wurde erst nach Stunden durch Deguggen entdeckt. Es versteckte sich in der Klasse Indexer.java in der Methode public int indexer(...). Hier war ein rekursiver Aufruf falsch implementiert, was zur 1000fachen indizierung einzelner Dateien f�hrte.

\textcolor{red}{HIER FEHLT NOCH MEHR}