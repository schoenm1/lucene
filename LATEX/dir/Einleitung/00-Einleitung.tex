\chapter{Einleitung}
\label{sec:Einleitung}
\section{Das Projekt}
%------------------------------------------------------------------------------------------------------------------------------
\subsection{Ausgangslage} \index{Ausgangslage}
\label{subsec:Ausganslage}
In den knapp 4 Jahren an der ZHAW haben sich verschiedenste Dateien angeh�uft. Die Thematiken �berschneiden die Module und sind somit nicht immer �ber eine �bersichtliche Ordnerstruktur auffindbar.\\
Die vielen Typen von Dokumenten (Office, pdf, txt, java...) vorhanden sind, wird durch die Betriebssystem integrierte Suche nicht immer das erwartete Resultat geliefert.
Alle diese Dokumente sollen �ber Lucene mit einem Index versehen und f�r eine effiziente Suche optimiert werden. So k�nnen unter anderem auch PDF inhaltlich indexiert werden, was bei der Search-Engine des Betriebssystems nicht funktioniert.

%------------------------------------------------------------------------------------------------------------------------------
\subsection{Ziel der Arbeit} \index{Ziel der Arbeit}
Das Ziel der Arbeit soll ein direkter Vergleich zwischen der Suchresultate von Lucene mit denjenigen des Betriebssystems stattfinden.\\
Aufgrund der subjektiven Sicht der suchenden Person soll schlussendlich entschieden werden, welche Search-Engine die besseren Suchresultate liefert. Es sollen mindestens zwei reale Begriffe gesucht werden und anhand dieser ein Fazit gezogen werden (eventuell mit Versbesserungsvorschl�gen f�r die Optimierung von Lucene).\\
F�r die Indexierung von PDF, welche nicht durch die Search-Engine des Betriebssystems bei der Indexierung eingeschlossen werden soll Lucene Abhilfe schaffen.
%------------------------------------------------------------------------------------------------------------------------------
\newpage
\subsection{Aufgabenstellung} \index{Aufgabenstellung}
\label{subsec:Ausgangslage}
\begin{enumerate}
\item Definierung der Suchkriterien �ber alle zu indexierenden Dateien
\item Erstellen einer einfachen Demo-Applikation in Java
\item Indexierung aller Schuldateien
\item Implementierung f�r die Indexierung von PDF. Wenn dies nicht direkt m�glich sein sollte in Lucene, werden PDF in Textdateien umgewandelt und dann indexiert
\item �berpr�fung der Suchresultate und Optimierung von Lucene
\item Reale Suche von mindestens zwei Begriffen in Lucene und direkter Verlgeich mit den Suchresultaten des Betriebssystemes
\item Fazit der Implementierung von Lucene. Sind die Suchresultate besser als diejenigen des Betriebssystemes?  Begr�ndung
\end{enumerate}

%------------------------------------------------------------------------------------------------------------------------------
\newpage
\subsection{Erwartetes Resultat} \label{subsec:erwartetesResultat}
\label{subsec:erwartetesResultat}
\begin{enumerate}
\item PDF sollen in den Suchresultaten mitber�cksichtigt werden
\item Die Eingabe der Suchbegriffe soll intuitiv und einfach gehalten werden
\item Die Qualit�t der Suchresultate von Lucene soll diejenige vom Betriebssystem �bertreffen. Falls dies nicht der Fall sein sollte, soll eine kurze Analyse aufzeigen, wieso das Betriebssystem bessere Resultate hervorbringen kann
\end{enumerate}

%------------------------------------------------------------------------------------------------------------------------------
\subsection{Geplante Termine}

\begin{table}[h!]
\begin{minipage}{10cm}
\begin{tabular}[t]{|l|l|} \hline
 \cellcolor{darkgrey} &  \cellcolor{darkgrey}  \\ 
\cellcolor{darkgrey} \multirow{-2}{3cm}{\textbf{Datum}} &
\cellcolor{darkgrey} \multirow{-2}{6cm}{\textbf{Beschreibung}}  \\  \cline{1-2}
14.03.2014 & Kick-Off Meeting   \\ \cline{1-2}
11.06.2014 & Abgabe der schriftlichen Arbeit (1 Woche vor Pr�sentation) \\ \cline{1-2}
18.06.2014 & Pr�sentation der Arbeit \\   \cline{1-2}
\end{tabular}
\caption{geplante Termine}
\label{tab:geplante_Termine}
\end{minipage}
\end{table}
