
%==============  N E W  ==== C H A P T E R ==============%
\chapter{Versionierung}
\label{chapter:Versionierung}

\begin{table}[h!]
\begin{minipage}{10cm}
\begin{tabular}[t]{|l|l|l|} \hline
\cellcolor{darkgrey} &  \cellcolor{darkgrey} &  \cellcolor{darkgrey}  \\ 
\cellcolor{darkgrey} \multirow{-2}{1.5cm}{\textbf{Version}} &
\cellcolor{darkgrey} \multirow{-2}{1.5cm}{\textbf{Datum}}  &
\cellcolor{darkgrey} \multirow{-2}{8cm}{\textbf{Beschreibung}} \\  \cline{1-3}
V0.1 & 18.03.2014 & Ersterstellung Dokument \\   \cline{1-3}
V0.2 & 11.05.2014 & Einarbeitung in Lucene, Installation Software, Lesen LiA \\   \cline{1-3}
V0.3 & 13.05.2014 & Buch LiA lesen / Coden \\   \cline{1-3}
V0.4 & 14.05.2014 & Doku schreiben, Query Optionen studieren \\   \cline{1-3}
V0.5 & 15.05.2014 & Doku LATEX, Query Optionen studieren \\   \cline{1-3}
V0.6 & 17.05.2014 & eingesetzte Hardware, Kapitel \nameref{chapter:Vorbereitung} \\   \cline{1-3}
V0.7 & 18.05.2014 & Kapitel \nameref{chapter:Vorbereitung}, Code Refactoring \\   \cline{1-3}
\end{tabular}
\caption{Versionierung Dokumentation}
\label{tab:Versionierung}
\end{minipage}
\end{table}







%==============  N E W  ==== C H A P T E R ==============%
\chapter{Aufw�nde}
\label{chapter:Aufw�nde}

\begin{table}[h!]
\begin{minipage}{10cm}
\begin{tabular}[t]{|l|l|l|} \hline
 \cellcolor{darkgrey} &  \cellcolor{darkgrey} &  \cellcolor{darkgrey}  \\ 
\cellcolor{darkgrey} \multirow{-2}{1.5cm}{\textbf{Datum}} & 
\cellcolor{darkgrey} \multirow{-2}{1.5cm}{\textbf{Zeit}}  &
\cellcolor{darkgrey} \multirow{-2}{8.5cm}{\textbf{Beschreibung}} \\  \cline{1-3}

 \multirow{3}{1cm}{18.03.2014} &  \multirow{3}{1cm}{1.25h} & Ersterstellung Dokument \\   
 & & -Download und Installation Lucene \\  
 & & -Erste Versuche mit Lucene\\   \cline{1-3}
 
 \multirow{2}{1cm}{26.03.2014} &  \multirow{2}{1cm}{3.75h} & Suche nach Thema f�r Eingabe EBS \\
  & & -Online Suche Funktionalit�ten Lucene\\   \cline{1-3}

 \multirow{4}{1cm}{11.05.2014} &  \multirow{4}{1cm}{4.25h} & Installation von Netbeans \\
& & -Update Eclipse\\
& & -Lucene in Action: Einlesen Kapitel 1\\
& & -1. Versuch Indexer und Searcher (gem�ss Beispielen LiA)\\ \cline{1-3}

12.05.2014 & 1.75h & siehe 11.05.2014\\   \cline{1-3}

 \multirow{3}{1cm}{13.05.2014} &  \multirow{3}{1cm}{7h} & -Indexierung erweitern auf alle Subdirectories des Root-Folders \\
 & & -Einschr�nken von File-Extensions:\\
 & & \ momentan nur Textdateien (.txt, .h, .c, .java)\\   \cline{1-3}

 \multirow{5}{1cm}{13.05.2014} &  \multirow{5}{1cm}{2.75h} & -Dokumentation Kapitel \ref{subsec:Ausgangslage} bis \ref{subsec:erwartetesResultat}\\
& & -Quellenverzeichnis erstellen\\
& & -Kapitel \refTC{subsec:eigesetzteSoftware}\\
& & -Kapitel \refTC{sec:LukeQuery}\\
& & -Einarbeiten in Query mit Lucene/Luke\\  \cline{1-3}

 \multirow{2}{1cm}{15.05.2014} &  \multirow{2}{1cm}{4.25h} & -Dokumentation von Word in LATEX umschreiben \\ 
 & & -Kapitel \refTC{sec:LukeQuery}\\ \cline{1-3}

 \multirow{2}{1cm}{17.05.2014} &  \multirow{2}{1cm}{2.25h} & -Kapitel \ref{chapter:Vorbereitung} mit teils Unterkapiteln \\ 
 & & -Kapitel \refTC{subsec:eigesetzteHardware}\\ \cline{1-3}

 \multirow{2}{1cm}{18.05.2014} &  \multirow{2}{1cm}{2.25h} & -Kapitel \ref{chapter:Vorbereitung} mit teils Unterkapiteln \\ 
 & & -Code Refactoring and cleaning\\ \cline{1-3}


\end{tabular}
\caption{Aufw�nde Seminararbeit}
\label{tab:Aufw�nde}
\end{minipage}
\end{table}
